\usepackage{amsmath}
\usepackage{ascmac}
\usepackage{amsthm}
\usepackage{tgpagella}
\usepackage{newpxmath}
\usepackage{algorithm}
\usepackage{url}
\usepackage{bm}
\usepackage{makeidx}
\usepackage{listings,jlisting}
\usepackage{color}
% copied from http://muscle-keisuke.hatenablog.com/entry/2016/02/11/195004
\lstset{%
  language={C++},
  basicstyle={\small},%
  identifierstyle={\small},%
  commentstyle={\small\itshape\color[rgb]{0,0.5,0}},%
  keywordstyle={\small\bfseries\color[rgb]{0,0,1}},%
  ndkeywordstyle={\small},%
  stringstyle={\small\ttfamily\color[rgb]{1,0,1}},
  frame={tb},
  breaklines=true,
  columns=[l]{fullflexible},%
  numbers=left,%
  xrightmargin=0zw,%
  xleftmargin=3zw,%
  numberstyle={\scriptsize},%
  stepnumber=1,
  numbersep=1zw,%
  lineskip=-0.5ex%
}

\newcommand{\Q}{\mathbb{Q}}
\newcommand{\Z}{\mathbb{Z}}
\newcommand{\legendre}[2]{\left(\frac{#1}{#2}\right)}
\newcommand{\level}[1]{(Lv.~#1)}
\newcommand{\finiteField}[1]{\mathrm{GF}(#1)}
\newcommand{\Frob}{\mathrm{Frob}}
\newcommand{\floor}[1]{\lfloor{#1}\rfloor}
\newcommand{\defineRuby}[2]{\emph{#1}\index{#2@#1}}
\newcommand{\tuple}[1]{\langle{#1}\rangle}

\theoremstyle{definition}
\newtheorem{theorem}{定理}[section]
\newtheorem{definition}[theorem]{定義}
\newtheorem{example}[theorem]{例}
\newtheorem{proposition}[theorem]{命題}
\newtheorem{corollary}[theorem]{系}
\newtheorem{remark}[theorem]{注意}
\renewcommand\proofname{\bf 証明}
